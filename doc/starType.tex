\documentclass[11pt]{article}
\usepackage[
        pdfencoding=auto,%
        pdftitle={starType},%
        pdfauthor={Edward F. Brown},%
        pdfstartview=FitV,%
        colorlinks=true,%
        linkcolor=blue,%
        citecolor=black, %
        urlcolor=blue]{hyperref}
\usepackage{txfonts}
\usepackage{longtable}
\usepackage{fancyvrb}
\usepackage{starType}

\DefineShortVerb{\|}
\fvset{numbers=right}

\begin{document}
    \thispagestyle{empty}
    \centerline{\rule[0.3ex]{\hsize}{2pt}}
    {\sffamily \noindent
    \LARGE starType\vspace{0.5ex}\\
    \Large Useful \LaTeX\ macros for stellar astrophysics}\\
    \centerline{\rule[0.6ex]{\hsize}{2pt}}
    \vspace{1.5em}

    \rmfamily
    
A common chore is the typesetting of units and various symbols. To help with this, I wrote a set of macros, \href{https://github.com/nworbde/starType}{\starType}.  You are welcome to use them or modify them to suit your needs.  

\section{Code Names}

\begin{center}
\begin{tabular}{ll}
\hline
command & produces\\
\hline\hline
|\flash| & \flash \\
|\kepler| & \kepler \\
|\nonsmoker| & \nonsmoker \\
|\mesa,\MESA| & \mesa \\
|\STERN| & \STERN \\
|\ADIPLS| & \ADIPLS \\
|\DSEP| & \DSEP \\
|\enzo| & \enzo \\
\hline
\end{tabular}
\end{center}

\section{Derivatives}

\begin{center}
    \renewcommand{\arraystretch}{1.5}
    \begin{tabular}{ll}
        \hline
        command & produces\\
        \hline\hline
        |\dif| & \dif \\
        |\Dif| & \Dif \\
        |\jac{a}{b}{c}{d}| & $\textstyle\jac{a}{b}{c}{d}$\\
        |\tderiv{a}{b}{c}| & $\textstyle\tderiv{a}{b}{c}$\\
        |\ddt{f}| & $\textstyle\ddt{a}$\\
        |\DDt{f}| & $\textstyle\DDt{f}$\\
        |\ddx{f}| & $\textstyle\ddx{f}$\\
        |\DDx{f}| & $\textstyle\DDx{f}$\\
        |\ddy{f}| & $\textstyle\ddy{f}$\\
        |\DDy{f}| & $\textstyle\DDy{f}$\\
        |\ddz{f}|  & $\textstyle\ddz{f}$\\
        |\DDz{f}| & $\textstyle\DDz{f}$\\
        \hline
    \end{tabular}
\end{center}

\section{Vectors}

\begin{center}
    \begin{tabular}{ll}
        \hline
        command & produces\\
        \hline\hline
        |\bvec{u}| & $\bvec{u}$\\
        |\grad f| & $\grad f$\\
        |\divr\bvec{u}| & $\divr\bvec{u}$\\
        |\curl\bvec{u}| & $\curl\bvec{u}$\\
        |\lap\phi| & $\lap\phi$\\
        |\btens{T}| & $\btens{T}$\\
        |\bvec{a}\vcross\bvec{b}| & $\bvec{a}\vcross\bvec{b}$\\
        |\bvec{a}\vdot\bvec{b}| & $\bvec{a}\vdot\bvec{b}$\\
        \hline
    \end{tabular}
\end{center}

\section{Nuclides}\label{s.nuclides.tex}

The \code{nuclides.tex} macros contain a list of all named elements.  Typeing `|\<element>|' produces the symbol of either the most common, or the longest-lived, isotope of that element.  To get a specific isotope, add the atomic number of the isotope in |[]|.  For example, |\carbon| produces \carbon, and |\carbon[13]| produces \carbon[13]; |\cadmium| produces \cadmium, whereas |\cadmium[116]| produces \cadmium[116]; and so on.
The symbols `|\neutron|' (alias `|\nt|') and `|\proton|' (alias `|\pt|') are also defined and produce `\neutron' and `\proton', respectively.

\section{Units}

To get scientific notation, type `|$3\ee{5}$|' to get $3\ee{5}$; alternatively, use `|\sci{3}{5}|' to get $\sci{3}{5}$.  To typeset a value with its unit, use the |\val| macro: for example, `|$\val{3}{\meter/\second}$|' produces $\val{3}{\meter/\second}$.  More complicated expressions are possible: for example,
\begin{center}
|$\val{\sci{2.0}{33}}{\ergspersecond}$| produces $\val{\sci{2.0}{33}}{\ergspersecond}$.
\end{center}
  For ranges of numbers, |\rng{2}{3}| produces \rng{2}{3}; |\rng[--]{2}{3}| produces \rng[--]{2}{3}. To put a range with a value, |\valrng{2}{3}{\meter/\second}| produces \valrng{2}{3}{\meter/\second} and |\valrng[--]{2}{3}{\meter/\second}| produces \valrng[--]{2}{3}{\meter/\second}.  Macros for the unit symbols are listed in the following table.

Note that more sophisticated packages, such as `SIunits' are available as part of a standard \LaTeX\ distribution.

Metric prefixes are defined.
\begin{center}
    \begin{tabular}{lll}
        \hline
        command & produces & meaning\\
        \hline\hline
|\yocto| & $\yocto$ & $10^{-24}$\\
|\zepto| & $\zepto$ & $10^{-21}$\\
|\atto| & $\atto$ & $10^{-18}$\\
|\femto| & $\femto$ & $10^{-15}$\\
|\pico| & $\pico$ & $10^{-12}$\\
|\nano| & $\nano$ & $10^{-9}$\\
|\micro| & $\micro$ & $ 10^{-6}$\\
|\milli| & $\milli$ & $10^{-3}$\\
|\centi| & $\centi$ & $10^{-2}$\\
|\deci| & $\deci$ & $10^{-1}$\\
%
|\deka| & $\deka$ & $10^{1}$\\
|\hecto| & $\hecto$ & $10^{2}$\\
|\kilo| & $\kilo$ & $10^{3}$\\
|\Mega| & $\Mega$ & $10^{6}$\\
|\Giga| & $\Giga$ & $10^{9}$\\
|\Tera| & $\Tera$ & $10^{12}$\\
|\Peta| & $\Peta$ & $10^{15}$\\
|\Exa| & $\Exa$ & $10^{18}$\\
|\Zetta| & $\Zetta$ & $10^{21}$\\
|\Yotta| & $\Yotta$ & $10^{24}$\\
        \hline
    \end{tabular}
\end{center}

A complete listing of the units are as follows.

    \begin{center}
    \begin{longtable}{lll}
        \hline
        command & produces & meaning\\
        \hline\hline
        |\meter| & $\meter$ & base units, mks\\
        |\kilogram| & $\kilogram$ & \\
        |\second| & $\second$ & \\
        |\Kelvin,\K| & $\Kelvin$ & degrees Kelvin \\
        |\cm| & $\cm$ &  base units, cgs\\
        |\gram| & $\gram$ & \\
        |\grampercc,\GramPerCc| & $\grampercc$ & mass density\\
        |\grampersquarecm,\GramPerSc,\columnunit| & $\grampersquarecm$ &  column depth\\
        |\dyne| & $\dyne$ & dyne\\
        |\erg,\ergs| & $\erg$ & ergs\\
        |\gauss| & $\gauss$ & gauss\\
        |\ergspersecond| & $\ergspersecond$ & \\
        |\ergspergram| & $\ergspergram$ & \\
        |\cgsflux| & $\cgsflux$ & cgs flux unit\\
        |\amu| & $\amu$ & atomic mass unit\\
        |\angstrom| & $\angstrom$ & Angstrom\\
        |\fermi| & $\fermi$ & fermi, aka femtometer\\
        |\eV| & $\eV$ & electron volt\\
        |\keV| & $\keV$ & \\ 
        |\MeV| & $\MeV$ & \\
        |\GeV| & $\GeV$ & \\
        |\MeVA| & $\MeVA$ &  MeV per nucleon\\
        |\GeVA| & $\GeVA$ & GeV per nucleon\\
        |\minute| & $\minute$ & minute\\
        |\hour| & $\hour$ & hour\\
        |\yr| & $\yr$ & year\\
        |\km| & $\km$ & kilometers\\
        |\Hz| & $\Hz$ & Hertz\\
        |\ksec| & $\ksec$ & kilosecond\\
        |\mol| & $\mol$ &  mole\\
        |\barn| & $\barn$  & barn\\
        |\Msun| & $\Msun$ & solar mass\\
        |\Lsun| & $\Lsun$ & solar luminosity\\
        |\Rsun| & $\Rsun$ & solar radius\\
        |\Myr| & $\Myr$ & \\
        |\Gyr| & $\Gyr$ & \\
        |\AU| & $\AU$ &  astronomical unit\\
        |\parsec| & $\parsec$ &  parsec\\
        |\kpc| & $\kpc$ &  kiloparsec\\
        |\Jansky| & $\Jansky$ &  Jansky\\
        |\mJy| & $\mJy$ &  micro Jansky\\
        |\Msunperyr| & $\Msunperyr$	& solar masses per year\\
        \hline
    \end{longtable}
\end{center}

\section{Symbols}

\begin{center}
\begin{longtable}{lll}
    \hline
    command & produces & meaning\\
    \hline\hline
|\abohr| & \abohr & Bohr radius \\
|\alphaF| & \alphaF & Fine structure \\
|\alphaMLT| & \alphaMLT &  mixing length parameter \\
|\alphasc| & \alphasc &  semiconvection efficiency parameter \\
|\alphath| & \alphath &  thermohaline efficiency parameter \\
|\chirho| & \chirho &  $(\partial\ln P/\partial\ln\rho)_T$ \\
|\chiT| & \chiT &  $(\partial\ln P/\partial\ln T)_{\rho}$ \\
|\CP| & \CP &  specific heat at constant pressure \\
|\cs| & \cs &  adiabatic sound speed \\
|\Dov| & \Dov &  overshoot diffusion coefficient \\
|\Dth| & \Dth &  thermohaline diffusion coefficient \\
|\EF| & \EF & Fermi energy \\
|\epsgrav| & \epsgrav &  gravitational heating rate \\
|\epsnu| & \epsnu &  neutrino losses \\
|\epsnuc| & \epsnuc &  nuclear heating rate \\
|\Fconv| & \Fconv &  convective flux \\
|\fov| & \fov &  convective overshoot parameter \\
|\Frad| & \Frad &  radiative flux \\
|\Gammaone| & \Gammaone &  $ (\partial\ln P/\partial \ln\rho)_S$ \\
|\Gammatwo| & \Gammatwo & $\left[1-(\partial\ln T/\partial\ln P)_S\right]^{-1}$ \\
|\Gammathree| & \Gammathree & $1+ (\partial\ln T/\partial\ln\rho)_S$\\
|\kB| & \kB &  Boltzmann constant \\
|\lambdaD| & \lambdaD & Debye length \\
|\Ledd| & \Ledd &  Eddington Luminosity \\
|\logg| & \logg &  log surface gravity \\
|\Lrad| & \Lrad &  radiative luminosity \\
|\Ma| & \Ma &  Mach number \\
|\mb| & \mb &  atomic mass unit \\
|\Mdot| & \Mdot &  mass-loss rate \\
|\me| & \me & electron mass \\
|\mn| & \mn & neutron mass \\
|\mpr| & \mpr & proton mass \\
|\NA| & \NA &  Avogadro number \\
|\nablaad| & \nablaad &  adiabatic temperature gradient \\
|\nablaL| & \nablaL &  Ledoux criterion \\
|\nablarad| & \nablarad &  radiative temperature gradient \\
|\nablaT| & \nablaT &  actual temperature gradient \\
|\nB| & \nB &  baryon density \\
|\Pc| & \Pc &  central pressure \\
|\pF| & \pF & Fermi momentum \\
|\Pgas| & \Pgas &  gas pressure \\
|\Prad| & \Prad &  radiation pressure \\
|\Rey| & \Rey &  Reynolds number \\
|\rhoc| & \rhoc &  central density \\
|\scaleheight| & \scaleheight &  pressure scale height \\
|\sigmaSB| & \sigmaSB &  Stefan-Boltzmann constant \\
|\Slamb| & \Slamb &  Lamb frequency \\
|\Tc| & \Tc &  central temperature \\
|\Teff,\teff| & \Teff &  effective temperature \\
|\tkh| & \tkh &  thermal (Kelvin-Helmholtz) timescale \\
\hline
\end{longtable}
\end{center}

\end{document}
