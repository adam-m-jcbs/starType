\documentclass[11pt]{article}
\usepackage[
        pdfencoding=auto,%
        pdftitle={starType},%
        pdfauthor={Edward F. Brown},%
        pdfstartview=FitV,%
        colorlinks=true,%
        linkcolor=blue,%
        citecolor=black, %
        urlcolor=blue]{hyperref}
\usepackage{txfonts}
\usepackage{longtable}
\usepackage{fancyvrb}

% units.tex
% Edward F Brown, Michigan State University
% 
% typesetting of dimensional values
% Features: proper spacing between value and unit; unit is set in correct type

% powers
\newcommand*{\ee}[1]{\ensuremath{\times 10^{#1}}}		% 3.0\ee{4} => 3.0\times 10^{4}
\newcommand*{\sci}[2]{\ensuremath{#1\ee{#2}}}			% \sci{3.0}{4} => 3.0\times 10^{4}
\newcommand{\power}[2]{\ensuremath{{#1}^{#2}}}		    % \power{2}{3} => 2^{3}

% spacing between value and unit; value is a thin space
\newcommand*{\unitskip}{\,}                             % change this to \; if you want a thicker space
\newcommand*{\usk}{\unitskip}
\newcommand*{\val}[2]{\ensuremath{#1\unitskip#2}}

% sometimes you want to input a range of numbers
% example: \rng{2}{3} puts 2 to 3; \rng[--]{2}{3} puts 2--3, where the two dashes are correctly typeset 
% as an en-dash rather than as a minus sign
\newcommand*{\rng}[3][~to~]{\ensuremath{#2\textrm{#1}#3}}
% this puts in a range with a unit
% example: \valrng[--]{2}{3}{\times \val{10^{3}}{\kilo\gram}}
\newcommand*{\valrng}[4][~to~]{\ensuremath{\left(#2\textrm{#1}#3\right)#4}}


% units should be in upright (roman) font
\newcommand*{\unitstyle}[1]{\mathrm{#1}}

% pre-defined macros for common units
% prefixes
\newcommand*{\yocto}{\unitstyle{y}}		% 10^{-24}
\newcommand*{\zepto}{\unitstyle{z}}		% 10^{-21}
\newcommand*{\atto}{\unitstyle{a}}		% 10^{-18}
\newcommand*{\femto}{\unitstyle{f}}		% 10^{-15}
\newcommand*{\pico}{\unitstyle{p}}		% 10^{-12}
\newcommand*{\nano}{\unitstyle{n}}		% 10^{-9}
\newcommand*{\micro}{\unitstyle{\mu}}	% 10^{-6}
\newcommand*{\milli}{\unitstyle{m}}		% 10^{-3}
\newcommand*{\centi}{\unitstyle{c}}		% 10^{-2}
\newcommand*{\deci}{\unitstyle{d}}		% 10^{-1}
%
\newcommand*{\deka}{\unitstyle{da}}		% 10^{1}
\newcommand*{\hecto}{\unitstyle{h}}		% 10^{2}
\newcommand*{\kilo}{\unitstyle{k}}		% 10^{3}
\newcommand*{\Mega}{\unitstyle{M}}		% 10^{6}
\newcommand*{\Giga}{\unitstyle{G}}		% 10^{9}
\newcommand*{\Tera}{\unitstyle{T}}		% 10^{12}
\newcommand*{\Peta}{\unitstyle{P}}		% 10^{15}
\newcommand*{\Exa}{\unitstyle{E}}		% 10^{18}
\newcommand*{\Zetta}{\unitstyle{Z}}		% 10^{21}
\newcommand*{\Yotta}{\unitstyle{Y}}		% 10^{24}

% base units, mks
\newcommand*{\meter}{\unitstyle{m}}
\newcommand*{\kilogram}{\kilo\gram}
\newcommand*{\second}{\unitstyle{s}}

\newcommand*{\Kelvin}{\unitstyle{K}}
\newcommand*{\K}{\Kelvin}  %degrees Kelvin

% base units, cgs
\newcommand*{\cm}{\centi\meter}
\newcommand*{\gram}{\unitstyle{g}}

% derived units
\newcommand*{\grampercc}{\gram\unitskip\power{\cm}{-3}} %mass density
\newcommand*{\grampersquarecm}{\gram\unitskip\power{\cm}{-2}} %column depth
\newcommand*{\GramPerCc}{\grampercc}
\newcommand*{\GramPerSc}{\grampersquarecm}
\newcommand*{\columnunit}{\grampersquarecm}
\newcommand*{\dyne}{\unitstyle{dyn}}            %dyne
\newcommand*{\erg}{\unitstyle{erg}}             %ergs
\newcommand*{\ergs}{\erg}
\newcommand*{\gauss}{\unitstyle{G}}             %gauss
\newcommand*{\ergspersecond}{\erg\unitskip\power{\second}{-1}}
\newcommand*{\ergspergram}{\erg\unitskip\power{\gram}{-1}}
\newcommand*{\cgsflux}{\erg\unitskip\power{\cm}{-2}\unitskip\power{\second}{-1}}

% Nuclear and atomic units
\newcommand*{\amu}{\unitstyle{u}}               %atomic mass unit
\newcommand*{\angstrom}{\mbox{\AA}}             %Angstrom
\newcommand*{\fermi}{\femto\meter}              %fermi
\newcommand*{\eV}{\unitstyle{eV}}               %eV
\newcommand*{\keV}{\kilo\eV}                    %keV
\newcommand*{\MeV}{\Mega\eV}                    %MeV
\newcommand*{\GeV}{\Giga\eV}                    %GeV
% Beam units
\newcommand*{\MeVA}{\MeV/A}                     % MeV per nucleon
\newcommand*{\GeVA}{\GeV/A}                     % GeV per nucleon

% misc. units
\newcommand*{\minute}{\unitstyle{min}}          %minute
\newcommand*{\hour}{\unitstyle{hr}}             %hour
\newcommand*{\yr}{\unitstyle{yr}}               %year
\newcommand*{\km}{\kilo\meter}                  %kilometers
\newcommand*{\Hz}{\unitstyle{Hz}}               %Hertz
\newcommand*{\ksec}{\kilo\second}               %kilosecond
\newcommand*{\mol}{\unitstyle{mol}}             % mole
\newcommand*{\barn}{\ensuremath{\mathrm{b}}}    %barn

% solar and astronomical units
\newcommand*{\Msun}{\ensuremath{M_\odot}}
\newcommand*{\Lsun}{\ensuremath{L_\odot}}
\newcommand*{\Rsun}{\ensuremath{R_\odot}}
\newcommand*{\Myr}{\Mega\yr}
\newcommand*{\Gyr}{\Giga\yr}
\newcommand*{\AU}{\unitstyle{AU}}               % astronomical unit
\newcommand*{\parsec}{\unitstyle{pc}}           % parsec
\newcommand*{\kpc}{\kilo\parsec}                % kiloparsec
\newcommand*{\Jansky}{\unitstyle{Jy}}           % Jansky
\newcommand*{\mJy}{\unitstyle{\mu Jy}}          % micro Jansky
\newcommand*{\Msunperyr}{\Msun\unitskip\power{\yr}{-1}}	% solar masses per year

% nuclides.tex
% Edward F. Brown, Michigan State University
%
% input file with macros for nuclides

% base command
\newcommand{\nuclei}[2]{\ensuremath{\mathrm{^{#1}#2}}}

% nuclides, with most highest abundance or longest half-life as default
% for example, \carbon produces ^{12}C, \carbon[13] produces ^{13}C
%
\newcommand{\neutron}{\ensuremath{\mathrm{n}}}
\newcommand{\nt}{\neutron}
\newcommand{\proton}{\ensuremath{\mathrm{p}}}
\newcommand{\pt}{\proton}
\newcommand{\hydrogen}[1][1]{\nuclei{#1}{H}}
\newcommand{\helium}[1][4]{\nuclei{#1}{He}}
\newcommand{\lithium}[1][7]{\nuclei{#1}{Li}}
\newcommand{\beryllium}[1][9]{\nuclei{#1}{Be}}
\newcommand{\boron}[1][11]{\nuclei{#1}{B}}
\newcommand{\carbon}[1][12]{\nuclei{#1}{C}}
\newcommand{\nitrogen}[1][14]{\nuclei{#1}{N}}
\newcommand{\oxygen}[1][16]{\nuclei{#1}{O}}
\newcommand{\fluorine}[1][19]{\nuclei{#1}{F}}
\newcommand{\neon}[1][20]{\nuclei{#1}{Ne}}
\newcommand{\sodium}[1][23]{\nuclei{#1}{Na}}
\newcommand{\magnesium}[1][24]{\nuclei{#1}{Mg}}
\newcommand{\aluminum}[1][27]{\nuclei{#1}{Al}}
\newcommand{\silicon}[1][28]{\nuclei{#1}{Si}}
\newcommand{\phosphorus}[1][31]{\nuclei{#1}{P}}
\newcommand{\sulfur}[1][32]{\nuclei{#1}{S}}
\newcommand{\chlorine}[1][35]{\nuclei{#1}{Cl}}
\newcommand{\argon}[1][36]{\nuclei{#1}{Ar}}
\newcommand{\potassium}[1][39]{\nuclei{#1}{K}}
\newcommand{\calcium}[1][40]{\nuclei{#1}{Ca}}
\newcommand{\scandium}[1][45]{\nuclei{#1}{Sc}}
\newcommand{\titanium}[1][48]{\nuclei{#1}{Ti}}
\newcommand{\vanadium}[1][51]{\nuclei{#1}{V}}
\newcommand{\chromium}[1][52]{\nuclei{#1}{Cr}}
\newcommand{\manganese}[1][55]{\nuclei{#1}{Mn}}
\newcommand{\iron}[1][56]{\nuclei{#1}{Fe}}
\newcommand{\cobalt}[1][59]{\nuclei{#1}{Co}}
\newcommand{\nickel}[1][58]{\nuclei{#1}{Ni}}
\newcommand{\copper}[1][63]{\nuclei{#1}{Cu}}
\newcommand{\zinc}[1][64]{\nuclei{#1}{Zn}}
\newcommand{\gallium}[1][69]{\nuclei{#1}{Ga}}
\newcommand{\germanium}[1][74]{\nuclei{#1}{Ge}}
\newcommand{\arsenic}[1][75]{\nuclei{#1}{As}}
\newcommand{\selenium}[1][80]{\nuclei{#1}{Se}}
\newcommand{\bromine}[1][79]{\nuclei{#1}{Br}}
\newcommand{\krypton}[1][84]{\nuclei{#1}{Kr}}
\newcommand{\rubidium}[1][85]{\nuclei{#1}{Rb}}
\newcommand{\strontium}[1][88]{\nuclei{#1}{Sr}}
\newcommand{\yttrium}[1][89]{\nuclei{#1}{Y}}
\newcommand{\zirconium}[1][94]{\nuclei{#1}{Zr}}
\newcommand{\niobium}[1][93]{\nuclei{#1}{Nb}}
\newcommand{\molybdenum}[1][98]{\nuclei{#1}{Mo}}
\newcommand{\technetium}[1][97]{\nuclei{#1}{Tc}}
\newcommand{\ruthenium}[1][102]{\nuclei{#1}{Ru}}
\newcommand{\rhodium}[1][103]{\nuclei{#1}{Rh }}
\newcommand{\palladium}[1][106]{\nuclei{#1}{Pd}}
\newcommand{\silver}[1][107]{\nuclei{#1}{Ag}}
\newcommand{\cadmium}[1][114]{\nuclei{#1}{Cd}}
\newcommand{\indium}[1][115]{\nuclei{#1}{In}}
\newcommand{\tin}[1][120]{\nuclei{#1}{Sn}}
\newcommand{\antimony}[1][121]{\nuclei{#1}{Sb}}
\newcommand{\tellurium}[1][130]{\nuclei{#1}{Te}}
\newcommand{\iodine}[1][127]{\nuclei{#1}{I}}
\newcommand{\xenon}[1][132]{\nuclei{#1}{Xe}}
\newcommand{\cesium}[1][133]{\nuclei{#1}{Cs}}
\newcommand{\barium}[1][138]{\nuclei{#1}{Ba}}
\newcommand{\lanthanum}[1][139]{\nuclei{#1}{La}}
\newcommand{\cerium}[1][140]{\nuclei{#1}{Ce}}
\newcommand{\praseodymium}[1][141]{\nuclei{#1}{Pr}}
\newcommand{\neodymium}[1][142]{\nuclei{#1}{Nd}}
\newcommand{\promethium}[1][147]{\nuclei{#1}{Pm}}
\newcommand{\samarium}[1][152]{\nuclei{#1}{Sm}}
\newcommand{\europium}[1][153]{\nuclei{#1}{Eu}}
\newcommand{\gadolinium}[1][158]{\nuclei{#1}{Gd}}
\newcommand{\terbium}[1][159]{\nuclei{#1}{Tb}}
\newcommand{\dysprosium}[1][164]{\nuclei{#1}{Dy}}
\newcommand{\holmium}[1][165]{\nuclei{#1}{Ho}}
\newcommand{\erbium}[1][168]{\nuclei{#1}{Er}}
\newcommand{\thulium}[1][169]{\nuclei{#1}{Tm}}
\newcommand{\ytterbium}[1][174]{\nuclei{#1}{Yb}}
\newcommand{\lutetium}[1][175]{\nuclei{#1}{Lu}}
\newcommand{\hafnium}[1][180]{\nuclei{#1}{Hf}}
\newcommand{\tantalum}[1][180]{\nuclei{#1}{Ta}}
\newcommand{\tungsten}[1][184]{\nuclei{#1}{W}}
\newcommand{\rhenium}[1][187]{\nuclei{#1}{Re}}
\newcommand{\osmium}[1][192]{\nuclei{#1}{Os}}
\newcommand{\iridium}[1][193]{\nuclei{#1}{Ir}}
\newcommand{\platnium}[1][195]{\nuclei{#1}{Pt}}
\newcommand{\gold}[1][197]{\nuclei{#1}{Au}}
\newcommand{\mercury}[1][202]{\nuclei{#1}{Hg}}
\newcommand{\thallium}[1][205]{\nuclei{#1}{Tl}}
\newcommand{\lead}[1][208]{\nuclei{#1}{Pb}}
\newcommand{\bisumth}[1][209]{\nuclei{#1}{Bi}}
\newcommand{\polonium}[1][210]{\nuclei{#1}{Po}}
\newcommand{\astatine}[1][210]{\nuclei{#1}{At}}
\newcommand{\radon}[1][222]{\nuclei{#1}{Rn}}
\newcommand{\francium}[1][223]{\nuclei{#1}{Fr}}
\newcommand{\radium}[1][226]{\nuclei{#1}{Ra}}
\newcommand{\actinium}[1][227]{\nuclei{#1}{Ac}}
\newcommand{\thorium}[1][232]{\nuclei{#1}{Th}}
\newcommand{\protactinium}[1][231]{\nuclei{#1}{Pa}}
\newcommand{\uranium}[1][238]{\nuclei{#1}{U}}
\newcommand{\neptunium}[1][237]{\nuclei{#1}{Np}}
\newcommand{\plutonium}[1][244]{\nuclei{#1}{Pu}}
\newcommand{\americium}[1][243]{\nuclei{#1}{Am}}
\newcommand{\curium}[1][247]{\nuclei{#1}{Cm}}
\newcommand{\berkelium}[1][247]{\nuclei{#1}{Bk}}
\newcommand{\californium}[1][251]{\nuclei{#1}{Cf}}
\newcommand{\einsteinium}[1][252]{\nuclei{#1}{Es}}
\newcommand{\fermium}[1][257]{\nuclei{#1}{Fm}}
\newcommand{\mendelevium}[1][258]{\nuclei{#1}{Md}}
\newcommand{\nobelium}[1][259]{\nuclei{#1}{No}}
\newcommand{\lawrencium}[1][262]{\nuclei{#1}{Lr}}
\newcommand{\rutherfordium}[1][261]{\nuclei{#1}{Rf}}
\newcommand{\dubnium}[1][268]{\nuclei{#1}{Db}}
\newcommand{\seaborgium}[1][271]{\nuclei{#1}{Sg}}
\newcommand{\bohrium}[1][274]{\nuclei{#1}{Bh}}
\newcommand{\hassium}[1][270]{\nuclei{#1}{Hs}}
\newcommand{\meitnerium}[1][278]{\nuclei{#1}{Mt}}
\newcommand{\darmstadtium}[1][281]{\nuclei{#1}{Ds}}
\newcommand{\roentgenium}[1][281]{\nuclei{#1}{Rg}}
\newcommand{\copernicum}[1][285]{\nuclei{#1}{Cn}}

% Features: proper spacing between value and unit; unit is set in correct type
% code.tex
% Edward F Brown, Michigan State University
% 
% Macros for commonly used codes
% 
\newcommand{\code}[1]{\texttt{#1}}
\newcommand{\flash}{FLASH}
\newcommand{\kepler}{KEPLER}
\newcommand{\nonsmoker}{NON-SMOKER}
\newcommand{\mesa}{\code{MESA}}
\newcommand{\MESA}{\mesa}
\newcommand{\STERN}{STERN}
\newcommand{\ADIPLS}{\code{ADIPLS}}
\newcommand{\DSEP}{DSEP}
\newcommand{\enzo}{\code{ENZO}}

% symbols.tex
% Edward Brown, Michigan State University
%
% symbols for commonly used expressions in stellar astrophysics.


\newcommand*{\epsnuc}{\ensuremath{\epsilon_{\mathrm{nuc}}}}	% nuclear heating rate
\newcommand*{\epsgrav}{\ensuremath{\epsilon_{\mathrm{grav}}}} % gravitational heating rate
\newcommand*{\epsnu}{\ensuremath{\epsilon_{\mathrm{\nu}}}} % neutrino losses
\newcommand*{\Teff}{\ensuremath{T_{\!\mathrm{eff}}}}	% effective temperature
\newcommand*{\teff}{\Teff}
\newcommand*{\Ledd}{\ensuremath{L_{\mathrm{Edd}}}} % Eddington Luminosity
\newcommand*{\logg}{\ensuremath{\log g}}	% log surface gravity
\newcommand*{\Tc}{\ensuremath{T_{\mathrm{\!c}}}} % central temperature
\newcommand*{\Pc}{\ensuremath{P_{\mathrm{\!c}}}} % central pressure
\newcommand*{\rhoc}{\ensuremath{\rho_{\mathrm{c}}}} % central density
\newcommand*{\CP}{\ensuremath{C_{\!P}}} % specific heat at constant pressure
\newcommand*{\Mdot}{\ensuremath{\dot{M}}} % mass-loss rate

% for correction between baryon densities and mass-energy densities
\newcommand*{\nB}{\ensuremath{n_{\mathrm{B}}}}	% baryon density

% thermodynamical Gammas
\newcommand*{\Gammaone}{\ensuremath{\Gamma_{\!1}}} % $ (\partial\ln P/\partial \ln\rho)_S$
\newcommand*{\Gammatwo}{\ensuremath{\Gamma_{\!2}}}
\newcommand*{\Gammatwo}{\ensuremath{\Gamma_{\!3}}}

% stellar gradients
\newcommand*{\alphaMLT}{\ensuremath{\alpha_{\mathrm{MLT}}}}	% mixing length parameter
\newcommand*{\chirho}{\ensuremath{\chi_{\rho}}}	% $(\partial\ln P/\partial\ln\rho)_T$
\newcommand*{\chiT}{\ensuremath{\chi_{\raisebox{-2pt}{$\scriptstyle T$}}}}	% $(\partial\ln P/\partial\ln T)_{\rho}$
\newcommand*{\Dov}{\ensuremath{D_{\mathrm{ov}}}}	% overshoot diffusion coefficient
\newcommand*{\nablaad}{\ensuremath{\nabla_{\!\mathrm{ad}}}}	% adiabatic temperature gradient
\newcommand*{\nablarad}{\ensuremath{\nabla_{\!\mathrm{rad}}}}	% radiative temperature gradient
\newcommand*{\nablaT}{\ensuremath{\nabla_{\!T}}}	% actual temperature gradient
\newcommand*{\nablaL}{\ensuremath{\nabla_{\mathrm{\!L}}}}	% Ledoux criterion
\newcommand*{\scaleheight}{\ensuremath{\lambda_P}}	% pressure scale height
\newcommand*{\Pgas}{\ensuremath{P_{\!\!\mathrm{gas}}}}	% gas pressure

% more symbols for radiation and gas pressures
\newcommand*{\Prad}{\ensuremath{P_{\!\!\mathrm{rad}}}}	% radiation pressure
\newcommand*{\Lrad}{\ensuremath{L_{\mathrm{rad}}}}	% radiative luminosity
\newcommand*{\Fconv}{\ensuremath{F_{\!\mathrm{conv}}}}		% convective flux
\newcommand*{\Frad}{\ensuremath{F_{\!\mathrm{rad}}}}	% radiative flux

% mixing symbols
\newcommand*{\alphasc}{\ensuremath{\alpha_{\mathrm{sc}}}} % semiconvection efficiency parameter
\newcommand*{\alphath}{\ensuremath{\alpha_{\mathrm{th}}}} % thermohaline efficiency parameter
\newcommand*{\Dth}{\ensuremath{D_{\mathrm{th}}}} % thermohaline diffusion coefficient
\newcommand*{\fov}{\ensuremath{f_{\mathrm{ov}}}} % convective overshoot parameter

% physical constants
\newcommand*{\kB}{\ensuremath{k_\mathrm{B}}} % Boltzmann constant
\newcommand*{\NA}{\ensuremath{N_\mathrm{\!A}}} % Avogadro number
\newcommand*{\mb}{\ensuremath{m_\mathrm{u}}} % atomic mass unit
\newcommand*{\sigmaSB}{\ensuremath{\sigma_\mathrm{\!SB}}} % Stefan-Boltzmann constant
\newcommand*{\pF}{\ensuremath{p_\mathrm{F}}} %Fermi momentum
\newcommand*{\EF}{\ensuremath{E_\mathrm{F}}} %Fermi energy
\newcommand*{\me}{\ensuremath{m_\mathrm{e}}} %electron mass
\newcommand*{\mpr}{\ensuremath{m_\mathrm{p}}} %proton mass
\newcommand*{\mn}{\ensuremath{m_\mathrm{n}}} %neutron mass
\newcommand*{\alphaF}{\ensuremath{\alpha_\mathrm{F}}} %Fine structure
\newcommand*{\lambdaD}{\ensuremath{\lambda_{\mathrm{D}}}} %Debye length
\newcommand*{\abohr}{\ensuremath{a_{\mathrm{B}}}} %Bohr radius


% stellar parameters
\newcommand*{\tkh}{\ensuremath{\tau_{\mathrm{KH}}}} % thermal (Kelvin-Helmholtz) timescale

% asteroseismology
\newcommand*{\cs}{\ensuremath{c_{\rm s}}} % adiabatic sound speed
\newcommand*{\Slamb}{\ensuremath{S_{\!\ell}}} % Lamb frequency

% fluids, dimensionless numbers
\newcommand*{\fluidnumber}[1]{\ensuremath{\mathrm{#1}}}
\newcommand*{\Ma}{\fluidnumber{Ma}}	% Mach number
\newcommand*{\Rey}{\fluidnumber{Re}}	% Reynolds number

% vectors.tex
% Edward Brown, Michigan State University
%
% Some basic vector operators
%
% Requires package bm.sty
%

\RequirePackage{bm}
\newcommand{\bvec}[1]{\ensuremath{\boldsymbol{#1}}} %boldface vector style
\newcommand{\grad}{\bvec{\nabla}} %gradient
\newcommand{\divr}{\nabla \cdot} %divergence
\newcommand{\curl}{\bvec{\nabla \times}} %curl
\newcommand{\lap}{\ensuremath{\nabla^2}} %Laplacian
\newcommand{\btens}[1]{\ensuremath{\boldsymbol{\mathsf{#1}}}}
\newcommand{\vcross}{\bvec{\times}}
\newcommand{\vdot}{\bvec{\cdot}}

% derivatives.tex
% Edward F Brown, Michigan State University
% 
% typesetting of common derivatives

\newcommand{\dif}{\ensuremath{\mathrm{d}}}  % differential operator, roman typeface
\newcommand*{\Dif}{\ensuremath{\mathrm{D}}} % lagrangian differential operator
\newcommand*{\jac}[4]{\ensuremath{\frac{\partial(#1,#2)}{\partial(#3,#4)}}} %jacobian
\newcommand*{\tderiv}[3]{\ensuremath{\left(\frac{\partial #1}{\partial #2}\right)_{#3}}} %thermodynamic derivative

%derivatives
\newcommand{\ddt}[1]{\frac{\partial #1}{\partial t}}    % partial time derivative 
\newcommand{\DDt}[1]{\frac{\dif #1}{\dif t}}            % total time derivative
\newcommand{\ddx}[1]{\frac{\partial #1}{\partial x}}    % partial derivative wrt x 
\newcommand{\ddy}[1]{\frac{\partial #1}{\partial y}}    % partial derivative wrt y 
\newcommand{\DDy}[1]{\frac{\dif #1}{\dif y}}            % total derivative wrt y
\newcommand{\ddz}[1]{\frac{\partial #1}{\partial z}}    % partial derivative wrt z 

\newcommand*{\starType}{\code{starType}}

\DefineShortVerb{\|}
\fvset{numbers=right}

\begin{document}
    \thispagestyle{empty}
    \centerline{\rule[0.3ex]{\hsize}{2pt}}
    {\sffamily \noindent
    \LARGE starType\vspace{0.5ex}\\
    \Large Useful \LaTeX\ macros for stellar astrophysics}\\
    \centerline{\rule[0.6ex]{\hsize}{2pt}}
    \vspace{1.5em}

    \rmfamily
    
A common chore is the typesetting of units and various symbols. To help with this, I wrote a set of macros, \href{https://github.com/nworbde/starType}{\starType}.  You are welcome to use them or modify them to suit your needs.  

\section{Code Names}

\begin{center}
\begin{tabular}{ll}
\hline
command & produces\\
\hline\hline
|\flash| & \flash \\
|\kepler| & \kepler \\
|\nonsmoker| & \nonsmoker \\
|\mesa,\MESA| & \mesa \\
|\STERN| & \STERN \\
|\ADIPLS| & \ADIPLS \\
|\DSEP| & \DSEP \\
|\enzo| & \enzo \\
\hline
\end{tabular}
\end{center}

\section{Derivatives}

\begin{center}
    \renewcommand{\arraystretch}{1.5}
    \begin{tabular}{ll}
        \hline
        command & produces\\
        \hline\hline
        |\dif| & \dif \\
        |\Dif| & \Dif \\
        |\jac{a}{b}{c}{d}| & $\textstyle\jac{a}{b}{c}{d}$\\
        |\tderiv{a}{b}{c}| & $\textstyle\tderiv{a}{b}{c}$\\
        |\ddt{f}| & $\textstyle\ddt{a}$\\
        |\DDt{f}| & $\textstyle\DDt{f}$\\
        |\ddx{f}| & $\textstyle\ddx{f}$\\
        |\ddy{f}| & $\textstyle\ddy{f}$\\
        |\DDy{f}| & $\textstyle\DDy{f}$\\
        |\ddz{f}|  & $\textstyle\ddz{f}$\\
        \hline
    \end{tabular}
\end{center}

\section{Vectors}

\begin{center}
    \begin{tabular}{ll}
        \hline
        command & produces\\
        \hline\hline
        |\bvec{u}| & $\bvec{u}$\\
        |\grad f| & $\grad f$\\
        |\divr\bvec{u}| & $\divr\bvec{u}$\\
        |\curl\bvec{u}| & $\curl\bvec{u}$\\
        |\lap\phi| & $\lap\phi$\\
        |\btens{T}| & $\btens{T}$\\
        |\bvec{a}\vcross\bvec{b}| & $\bvec{a}\vcross\bvec{b}$\\
        |\bvec{a}\vdot\bvec{b}| & $\bvec{a}\vdot\bvec{b}$\\
        \hline
    \end{tabular}
\end{center}

\section{Nuclides}\label{s.nuclides.tex}

The \code{nuclides.tex} macros contain a list of all named elements.  Typeing `|\<element>|' produces the symbol of either the most common, or the longest-lived, isotope of that element.  To get a specific isotope, add the atomic number of the isotope in |[]|.  For example, |\carbon| produces \carbon, and |\carbon[13]| produces \carbon[13]; |\cadmium| produces \cadmium, whereas |\cadmium[116]| produces \cadmium[116]; and so on.
The symbols `|\neutron|' (alias `|\nt|') and `|\proton|' (alias `|\pt|') are also defined and produce `\neutron' and `\proton', respectively.

\section{Units}

To get scientific notation, type `|$3\ee{5}$|' to get $3\ee{5}$; alternatively, use `|\sci{3}{5}|' to get $\sci{3}{5}$.  To typeset a value with its unit, use the |\val| macro: for example, `|$\val{3}{\meter/\second}$|' produces $\val{3}{\meter/\second}$.  More complicated expressions are possible: for example,
\begin{center}
|$\val{\sci{2.0}{33}}{\ergspersecond}$| produces $\val{\sci{2.0}{33}}{\ergspersecond}$.
\end{center}
  For ranges of numbers, |\rng{2}{3}| produces \rng{2}{3}; |\rng[--]{2}{3}| produces \rng[--]{2}{3}. To put a range with a value, |\valrng{2}{3}{\meter/\second}| produces \valrng{2}{3}{\meter/\second} and |\valrng[--]{2}{3}{\meter/\second}| produces \valrng[--]{2}{3}{\meter/\second}.  Macros for the unit symbols are listed in the following table.

Note that more sophisticated packages, such as `SIunits' are available as part of a standard \LaTeX\ distribution.

Metric prefixes are defined.
\begin{center}
    \begin{tabular}{lll}
        \hline
        command & produces & meaning\\
        \hline\hline
|\yocto| & $\yocto$ & $10^{-24}$\\
|\zepto| & $\zepto$ & $10^{-21}$\\
|\atto| & $\atto$ & $10^{-18}$\\
|\femto| & $\femto$ & $10^{-15}$\\
|\pico| & $\pico$ & $10^{-12}$\\
|\nano| & $\nano$ & $10^{-9}$\\
|\micro| & $\micro$ & $ 10^{-6}$\\
|\milli| & $\milli$ & $10^{-3}$\\
|\centi| & $\centi$ & $10^{-2}$\\
|\deci| & $\deci$ & $10^{-1}$\\
%
|\deka| & $\deka$ & $10^{1}$\\
|\hecto| & $\hecto$ & $10^{2}$\\
|\kilo| & $\kilo$ & $10^{3}$\\
|\Mega| & $\Mega$ & $10^{6}$\\
|\Giga| & $\Giga$ & $10^{9}$\\
|\Tera| & $\Tera$ & $10^{12}$\\
|\Peta| & $\Peta$ & $10^{15}$\\
|\Exa| & $\Exa$ & $10^{18}$\\
|\Zetta| & $\Zetta$ & $10^{21}$\\
|\Yotta| & $\Yotta$ & $10^{24}$\\
        \hline
    \end{tabular}
\end{center}

A complete listing of the units are as follows.

    \begin{center}
    \begin{longtable}{lll}
        \hline
        command & produces & meaning\\
        \hline\hline
        |\meter| & $\meter$ & base units, mks\\
        |\kilogram| & $\kilogram$ & \\
        |\second| & $\second$ & \\
        |\Kelvin,\K| & $\Kelvin$ & degrees Kelvin \\
        |\cm| & $\cm$ &  base units, cgs\\
        |\gram| & $\gram$ & \\
        |\grampercc,\GramPerCc| & $\grampercc$ & mass density\\
        |\grampersquarecm,\GramPerSc,\columnunit| & $\grampersquarecm$ &  column depth\\
        |\dyne| & $\dyne$ & dyne\\
        |\erg,\ergs| & $\erg$ & ergs\\
        |\gauss| & $\gauss$ & gauss\\
        |\ergspersecond| & $\ergspersecond$ & \\
        |\ergspergram| & $\ergspergram$ & \\
        |\cgsflux| & $\cgsflux$ & cgs flux unit\\
        |\amu| & $\amu$ & atomic mass unit\\
        |\angstrom| & $\angstrom$ & Angstrom\\
        |\fermi| & $\fermi$ & fermi, aka femtometer\\
        |\eV| & $\eV$ & electron volt\\
        |\keV| & $\keV$ & \\ 
        |\MeV| & $\MeV$ & \\
        |\GeV| & $\GeV$ & \\
        |\MeVA| & $\MeVA$ &  MeV per nucleon\\
        |\GeVA| & $\GeVA$ & GeV per nucleon\\
        |\minute| & $\minute$ & minute\\
        |\hour| & $\hour$ & hour\\
        |\yr| & $\yr$ & year\\
        |\km| & $\km$ & kilometers\\
        |\Hz| & $\Hz$ & Hertz\\
        |\ksec| & $\ksec$ & kilosecond\\
        |\mol| & $\mol$ &  mole\\
        |\barn| & $\barn$  & barn\\
        |\Msun| & $\Msun$ & solar mass\\
        |\Lsun| & $\Lsun$ & solar luminosity\\
        |\Rsun| & $\Rsun$ & solar radius\\
        |\Myr| & $\Myr$ & \\
        |\Gyr| & $\Gyr$ & \\
        |\AU| & $\AU$ &  astronomical unit\\
        |\parsec| & $\parsec$ &  parsec\\
        |\kpc| & $\kpc$ &  kiloparsec\\
        |\Jansky| & $\Jansky$ &  Jansky\\
        |\mJy| & $\mJy$ &  micro Jansky\\
        |\Msunperyr| & $\Msunperyr$	& solar masses per year\\
        \hline
    \end{longtable}
\end{center}

\section{Symbols}

\begin{center}
\begin{longtable}{lll}
    \hline
    command & produces & meaning\\
    \hline\hline
|\abohr| & \abohr & Bohr radius \\
|\alphaF| & \alphaF & Fine structure \\
|\alphaMLT| & \alphaMLT &  mixing length parameter \\
|\alphasc| & \alphasc &  semiconvection efficiency parameter \\
|\alphath| & \alphath &  thermohaline efficiency parameter \\
|\chirho| & \chirho &  $(\partial\ln P/\partial\ln\rho)_T$ \\
|\chiT| & \chiT &  $(\partial\ln P/\partial\ln T)_{\rho}$ \\
|\CP| & \CP &  specific heat at constant pressure \\
|\cs| & \cs &  adiabatic sound speed \\
|\Dov| & \Dov &  overshoot diffusion coefficient \\
|\Dth| & \Dth &  thermohaline diffusion coefficient \\
|\EF| & \EF & Fermi energy \\
|\epsgrav| & \epsgrav &  gravitational heating rate \\
|\epsnu| & \epsnu &  neutrino losses \\
|\epsnuc| & \epsnuc &  nuclear heating rate \\
|\Fconv| & \Fconv &  convective flux \\
|\fov| & \fov &  convective overshoot parameter \\
|\Frad| & \Frad &  radiative flux \\
|\Gammaone| & \Gammaone &  $ (\partial\ln P/\partial \ln\rho)_S$ \\
|\Gammatwo| & \Gammatwo & $\left[1-(\partial\ln T/\partial\ln P)_S\right]^{-1}$ \\
|\Gammathree| & \Gammathree & $1+ (\partial\ln T/\partial\ln\rho)_S$\\
|\kB| & \kB &  Boltzmann constant \\
|\lambdaD| & \lambdaD & Debye length \\
|\Ledd| & \Ledd &  Eddington Luminosity \\
|\logg| & \logg &  log surface gravity \\
|\Lrad| & \Lrad &  radiative luminosity \\
|\Ma| & \Ma &  Mach number \\
|\mb| & \mb &  atomic mass unit \\
|\Mdot| & \Mdot &  mass-loss rate \\
|\me| & \me & electron mass \\
|\mn| & \mn & neutron mass \\
|\mpr| & \mpr & proton mass \\
|\NA| & \NA &  Avogadro number \\
|\nablaad| & \nablaad &  adiabatic temperature gradient \\
|\nablaL| & \nablaL &  Ledoux criterion \\
|\nablarad| & \nablarad &  radiative temperature gradient \\
|\nablaT| & \nablaT &  actual temperature gradient \\
|\nB| & \nB &  baryon density \\
|\Pc| & \Pc &  central pressure \\
|\pF| & \pF & Fermi momentum \\
|\Pgas| & \Pgas &  gas pressure \\
|\Prad| & \Prad &  radiation pressure \\
|\Rey| & \Rey &  Reynolds number \\
|\rhoc| & \rhoc &  central density \\
|\scaleheight| & \scaleheight &  pressure scale height \\
|\sigmaSB| & \sigmaSB &  Stefan-Boltzmann constant \\
|\Slamb| & \Slamb &  Lamb frequency \\
|\Tc| & \Tc &  central temperature \\
|\Teff,\teff| & \Teff &  effective temperature \\
|\tkh| & \tkh &  thermal (Kelvin-Helmholtz) timescale \\
\hline
\end{longtable}
\end{center}

\end{document}
